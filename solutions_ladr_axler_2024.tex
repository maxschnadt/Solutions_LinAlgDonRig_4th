\documentclass[11pt, b5paper, draft, fleqn]{book}
\usepackage[inner=2.00cm, outer=2.0cm, top=2.0cm, bottom=2.0cm]{geometry}
%
%
\usepackage[english]{babel}
%
%
\usepackage{amsmath}
\usepackage{amsfonts}
%\usepackage{amssymb}
\usepackage{amsthm}

\theoremstyle{remark}
\newtheorem*{solution}{Solution}

\theoremstyle{definition}
\newtheorem{defn}{Definition}

%-----------------Grabkiste nach Beweisen-------%
\renewcommand{\qedsymbol}{\(\hfill\blacksquare\)}
%
%
\usepackage[T1]{fontenc}
%\usepackage{tgschola}
\usepackage{stickstootext}
\usepackage[stickstoo,vvarbb]{newtxmath}
%
%
\usepackage{graphicx}
\usepackage{color}
%
%
\usepackage{extramarks}

\usepackage{hyperref}
\hypersetup{
	colorlinks=true,
	linktoc=all,
	linkcolor=blue,
	pdftoolbar=true,
	bookmarks=true,
	bookmarksnumbered=true
}

\usepackage{fancyhdr}
\setlength{\headheight}{15.2pt}
\renewcommand{\headrulewidth}{0pt}
\renewcommand{\footrulewidth}{0pt}
\pagestyle{fancy}
%
%
\usepackage{enumitem}
\setlist[enumerate, 1]{label=\thesection.\arabic*}
\setlist[enumerate, 2]{label=(\arabic*)}
\setlist[enumerate, 3]{label=\arabic*}

\newlist{rome}{enumerate}{1}
\setlist[rome]{label=\Roman*.}

\newlist{alphbet}{enumerate}{1}
\setlist[alphbet]{label=\alph*.}

\begin{document}
\begin{titlepage}	
	\begin{center}
	
	\huge
	\textbf{Solutions Manual}
    
	\vspace{2em}
	\Large
	\textbf{Linear Algebra Done Right - 4th Edition}
    
	Sheldon Axler
	
	\vspace{2em}
	\large
	\textbf{This solution manual was created by the MathLearners study group.}
	
	\end{center}
\end{titlepage}

\fancyhf{}
\cleardoublepage
\pagenumbering{Roman}
\setcounter{page}{3}
\tableofcontents

\cleardoublepage

\fancyhf{}
\pagenumbering{arabic}
\setcounter{page}{5}
\rhead{\thepage}
\lhead{\firstleftmark}
\sloppy

\chapter{Vector Spaces}
\section{A: \(\mathbb{R}^n\) and \(\mathbb{C}^n\)}
\begin{enumerate}
    \item[4]
	Show that \(\lambda(\alpha + \beta) = \lambda\alpha + \lambda\beta\) for \(\lambda, \alpha, \beta \in \mathbb{C}\).
	\begin{proof}
		Assume \(\lambda, \alpha, \beta \in \mathbb{C}\). Thus \(\lambda = a+bi, \alpha = c+di\) and \(\beta = f+gi\). Thus \(\lambda(\alpha + \beta) = (a+bi)(c+di + f+gi)\). From this we get \(a(c+di + f+gi) + bi(c+di + f+gi) = (ac + adi + af+ agi) + (bci + bdi^2 +bfi + bgi^2)\). And finally \(\left[(ac - bd) + (ad + bc)i\right] + \left[(af - bg) + (ag + bf)i\right] = \lambda\alpha + \lambda\beta\). This completes our proof.
	\end{proof}
	\item[7] Show that \(\frac{-1+\sqrt{3}i}{2}\) is a cube root of 1 (meaning that its cube equals 1).
	\begin{proof}
		\begin{equation*}
		\begin{split}
			\left(\frac{-1 + \sqrt{3}i}{2}\right)^3 & = \frac{(-1 + \sqrt{3}i)^3}{2^3} \\
			& = \frac{(-1 + \sqrt{3}i) (-1 + \sqrt{3}i)^2}{8} \\
			& = \frac{(-1 + \sqrt{3}i)\left[(1 - 3)+(-\sqrt{3} - \sqrt{3})i\right]}{8} \\
			& = \frac{(-1 + \sqrt{3}i) (-2 + -2\sqrt{3}i)}{8} \\
			& = \frac{(2 + 6) + (2\sqrt{3}-2\sqrt{3})i}{8} \\
			& = \frac{8 + 0i}{8} = 1.
		\end{split}
		\end{equation*}
	\end{proof}
	\item[9] Find \(x \in \mathbb{R}^4\) such that \[(4, -3, 1, 7) + 2x = (5, 9, -6, 8).\]
	\begin{solution}
		\begin{equation*}
		\begin{split}
			(4, -3, 1, 7) + 2x & = (5, 9, -6, 8) \\
			2x & = (5, 9, -6, 8) - (4, -3, 1, 7) \\
			& = (1, 12, -7, 1) \\
			x & = (0.5, 6, -3.5, 0.5).
		\end{split}
		\end{equation*}
	\end{solution}
	\item[10] Explain why there does not exist \(\lambda \in \mathbb{C}\) such that \[\lambda(2 - 3i, 5 + 4i, -6 + 7i) = (12 - 5i, 7 + 22i, -32 - 9i).\]
	\begin{solution}
		For \(\lambda\) to exist, it would have to satisfy \(2 \lambda = 12\) and \(5 \lambda = 7\) (real part of the first two coordinates). But in the first case we get \(\lambda = 6\), and in the second case we get \(\lambda = 1.4\). Thus \(\lambda\) can't exist.
	\end{solution}
\end{enumerate}
\section{B: Definition of Vector Space}
\begin{enumerate}
	\item[1] Prove that \(-(-\mathbf{v}) = \mathbf{v}\).
	\begin{proof}
		We use direct proof. Thus let \(\mathbf{v} \in V\) and \(\mathbf{-v}\) be the additive inverse of \(\mathbf{v}\). Thus \(\mathbf{v}\) is the additive inverse of \(\mathbf{-v}\). Thus \(-(\mathbf{-v}) = \mathbf{v}\), which completes our proof.
	\end{proof}
	
	\item[2] Suppose \(\alpha \in \mathbb{F}, \mathbf{v} \in V\) and \(\alpha \mathbf{v} = \mathbf{0}\). Prove that \(\alpha = 0\) or \(\mathbf{v} = \mathbf{0}\).
	\begin{proof}
		We will use proof by contradiction. Thus assume \(\alpha \mathbf{v} = \mathbf{0}\), \(\alpha \ne 0\) and \(\mathbf{v} \ne \mathbf{0}\). Since \(\alpha \ne 0\), we can multiply both sides of \(\alpha \mathbf{v} = \mathbf{0}\) by \(\frac{1}{\alpha}\). From this we get
		\begin{equation*}
		\begin{split}
			\frac{\alpha}{\alpha} \mathbf{v} & = \frac{1}{\alpha} \mathbf{0} \\
			1 \mathbf{v} & = \frac{1}{\alpha} 0\mathbf{v} \\
			\mathbf{v} & = \frac{0}{\alpha} \mathbf{v} \\
			& = 0 \mathbf{v} \\
			& = \mathbf{0}.
		\end{split}
		\end{equation*}
		But this contradicts \(\mathbf{v} \ne \mathbf{0}\), which completes our proof.
	\end{proof}
	
	\item[4] The empty set is not a vector space. The empty set fails to satisfy only one of the requirements listend in the definition of a vector space. Which one?
	\begin{solution}
		Since the empty set \(\varnothing\) contains no elements, it fails the condition that \(0 \in \varnothing\).
	\end{solution}
	
	\item[5] Show that in the definition of a vector space, the additive inverse condition can be replaced with the condition that \[0 \mathbf{v} = \mathbf{0}\] for all \(\mathbf{v} \in V\).
	\begin{proof}
		We will show that the definitions can be replaced by showing their equivalency. As we've seen in the chapter, the condition \(\mathbf{v} + \mathbf{0} = \mathbf{v}\) implies \(0 \mathbf{v} = \mathbf{0}\). 
		
		For the converse, suppose \(0 \mathbf{v} = \mathbf{0}\). Adding \(\mathbf{v}\) to both sides, we get \(0 \mathbf{v} + \mathbf{v} = 0 \mathbf{v} + 1 \mathbf{v} = \mathbf{0} + \mathbf{v}\), and so \((0 +1) \mathbf{v} = 1 \mathbf{v} = \mathbf{v} = \mathbf{0} + \mathbf{v}\), which completes our proof.
	\end{proof}
	
	\item[7] Suppose \(\mathcal{S}\) is a nonempty set. Let \(V^{\mathcal{S}}\) denote the set of functions from \(\mathcal{S}\) to \(V\). Define a natural addition and scalar multiplication on \(V^{\mathcal{S}}\), and show that \(V^{\mathcal{S}}\) is a vector space with these definitions.
\end{enumerate}
\end{document}